% % \documentclass{beamer}
% \documentclass[10pt,UTF8]{ctexbeamer} 
% \usetheme{metropolis}    
% \usepackage{ctexart}       % Use metropolis theme

\documentclass[10pt,UTF8]{ctexbeamer} 
%默认字体12pt,编码使用UTF8,使用中文

\usetheme[progressbar=frametitle]{metropolis} 
%使用metropolis这一beamer主题
\title{Python}
\date{\today}
\author{快乐的老鼠宝宝}
\institute{长城学生网络}
\begin{document}
  \maketitle
  \section{First Section}
\begin{frame}{1. 环境安装}
不会有人还没装环境吧不是吧不是吧
\end{frame}
\begin{frame}{1.1.  pip}
整几个常用的package下来,顺便pypi是什么
\end{frame}
\begin{frame}{1.2.  import的用法}
直接import
\end{frame}
\begin{frame}{1.3. 基础语法}
数据类型
数据类型转换
类型提示系统
各种常见的方法,如replace,append,split等
从0开始计数
条件控制语句
文件读写
生成式和函数式(选讲)
面向对象(选讲)
\end{frame}
\begin{frame}{2. 分别各种场景下的实战}
1.4. 高性能计算
那您上Pypy啊
还嫌慢那别用Python啊
\end{frame}
\begin{frame}{2. 分别各种场景下的实战}
\end{frame}
\begin{frame}{2.1. 纯粹就做个计算器,算账用的}
(莽就完事了)
\end{frame}
\begin{frame}{2.2. 如果我是个自动化工作流爱好者呢?}
os, pywin32
\end{frame}
\begin{frame}{2.3. 爬虫小子}
requests, xml/lxml, json, regex, beautifulsoup
\end{frame}
\begin{frame}{2.4. 科学计算和绘图}
numpy, pandas, scipy, sympy, anaconda
\end{frame}
\begin{frame}{2.5. 想用python写GUI?}
嫌PyQt/WxWidget太难用?欢迎入坑Flet
\end{frame}
\begin{frame}{2.6. 也不是不能写服务端}
来我们看一下Django
\end{frame}
\begin{frame}{3. 还有}
    rdyd rpc
    \end{frame}
\end{document}