% % \documentclass{beamer}
% \documentclass[10pt,UTF8]{ctexbeamer} 
% \usetheme{metropolis}    
% \usepackage{ctexart}       % Use metropolis theme

\documentclass[10pt,UTF8]{ctexbeamer} 
%默认字体12pt,编码使用UTF8,使用中文

\usetheme[progressbar=frametitle]{metropolis} 
%使用metropolis这一beamer主题
\title{Python}
\date{\today}
\author{Matthias Vogelgesang}
\institute{Centre for Modern Beamer Themes}
\begin{document}
  \maketitle
  \section{First Section}
  \begin{frame}{First Frame}
1. 环境安装:不会有人还没装环境吧不是吧不是吧
1.1.  用pip整几个常用的package下来,顺便pypi是什么
1.2.  import的用法(基本)
1.3. 基础语法
数据类型
数据类型转换
类型提示系统
各种常见的方法,如replace,append,split等
从0开始计数
条件控制语句
文件读写
生成式和函数式(选讲)
面向对象(选讲)
1.4. 高性能计算
那您上Pypy啊
还嫌慢那别用Python啊

2.  分别各种场景下的实战
2.1. 纯粹就做个计算器,算账用的(莽就完事了)
2.2. 如果我是个自动化工作流爱好者呢?(os, pywin32)
2.3. 爬虫小子(requests, xml/lxml, json, regex, beautifulsoup)
2.4. 科学计算和绘图(numpy, pandas, scipy, sympy, anaconda)
2.5. 想用python写GUI?嫌PyQt/WxWidget太难用?欢迎入坑Flet
2.6. 也不是不能写服务端,来我们看一下Django
  \end{frame}
\end{document}